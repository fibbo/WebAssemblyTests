%! TEX program = lualatex
\documentclass[10pt, a4paper]{beamer} %handout fuer eine gedruckte version der slides
\usetheme{metropolis}

\usepackage{url}
\usepackage{polyglossia}
\setmainlanguage{english}

\usepackage{listings,xcolor}
\usepackage{graphicx}
\usepackage{nameref}
\usepackage{mathtools}
\usepackage{tikz}

\usepackage{booktabs}

\makeatletter
\newcommand*{\currentname}{\@currentlabelname}
\makeatother

\usepackage{fontspec}
\newfontfamily\Bera{Fira Code}[Scale=0.85]

\setmonofont[
  Contextuals={Alternate}
]{Fira Code}


\definecolor{pdfToolsBlue}{HTML}{0084C7}
\definecolor{pdfToolsDarkBlue}{HTML}{004569}
\definecolor{pdfToolsDarkerBlue}{HTML}{002A3F}
\definecolor{pdfToolsRenderingOrange}{HTML}{E78B00}

\definecolor{mDarkTeal}{HTML}{23373b}
\definecolor{mLightBrown}{HTML}{EB811B}
\definecolor{mDarkBlue}{HTML}{0F3470}
\definecolor{mDarkGrey}{HTML}{999999}
\definecolor{mLighterGrey}{HTML}{CCCCCC}
\definecolor{mLightGrey}{rgb}{0.95,0.95,0.95}
\definecolor{mLightBlue}{HTML}{212751}

\newcommand{\lb}[1]{{\color{pdfToolsRenderingOrange}#1}}

\newcommand{\bblock}[3]{
        \begin{block}{#1}
        #2
    \end{block}
}

\setbeamercolor{block title}{fg=pdfToolsDarkBlue}
\setbeamercolor{frametitle}{bg=pdfToolsBlue,fg=white}
\setbeamercolor{normal text}{fg=black}

\lstdefinestyle{JavaScript}{
  keywords={typeof, new, true, false, catch, function, return, null, catch, switch, var, if, in, while, do, else, case, break},
  keywordstyle=\color{blue}\bfseries,
  ndkeywords={class, export, boolean, throw, implements, import, this},
  ndkeywordstyle=\color{darkgray}\bfseries,
  identifierstyle=\color{black},
  sensitive=false,
  comment=[l]{//},
  morecomment=[s]{/*}{*/},
  commentstyle=\color{purple}\ttfamily,
  stringstyle=\color{red}\ttfamily,
  morestring=[b]',
  morestring=[b]"
}

\lstset{language=C++,
    backgroundcolor=\color{mLightGrey},
    frame=single,
    rulecolor=\color{mLightGrey}, % make frame "invisible"
    basicstyle=\Bera\tiny,
    keywordstyle=\bfseries\color{mDarkBlue},
    commentstyle=\color{mDarkGrey}\itshape,
    captionpos=t,
    emphstyle=\ttb\color{mDarkBlue},    % Custom highlighting style
    stringstyle=\color{mLightBrown},
    tabsize=2,
    numberbychapter=false,
    showstringspaces=false,
    breaklines=true,
    morekeywords={assert, break, class, continue, else, except, for, if, return, try, while}
}
%
\makeatletter
\setbeamertemplate{footline}{%
\leavevmode
\vbox{\begin{beamercolorbox}[dp=1.25ex,ht=2.75ex]{fg=black}%
  \hspace*{1em}\insertsectionhead%
  \ifx\insertsubsectionhead\@empty\relax\else\hspace*{\fill}\insertsubsectionhead\hspace*{2ex}\fi
  \end{beamercolorbox}%
  }%
}
\makeatother


\newcommand{\itembullet}{
    \tikz\draw[pdfToolsDarkerBlue,fill=pdfToolsDarkerBlue] (0,0) circle (.4ex);
}

\AtBeginSection{}

\setbeamertemplate{section in toc}{%
  {\color{mDarkTeal}\rule[0.3ex]{3pt}{3pt}}~\inserttocsection\par}
\setbeamertemplate{subsection in toc}{%
  \hspace{1.2em}{\color{mLightBrown}\rule[0.3ex]{3pt}{3pt}}~\inserttocsubsection\par}


\setbeamertemplate{itemize item}{\itembullet}
\setlength{\parskip}{4ex}

\title % (optional, only for long titles)
{Dynamic Loading of Modules in WASM/Emscripten}
\author % (optional, for multiple authors)
{Philipp Gloor}
\date{}
\titlegraphic{\raggedleft\includegraphics[width=0.2\textwidth]{img/logo-pdftools-c-200}}
\begin{document}
\begin{frame}
\titlepage
\end{frame}

{
  \setlength{\parskip}{0ex}
\begin{frame}{Table of Contents}
    \tableofcontents
  \end{frame}
}

\section{Interaction JavaScript <-> Emscripten}
\begin{frame}{Passing values: Reference or Value}
\begin{itemize}
  \item Passing objects created in JavaScript into Emscripten copies the objects
  \item Passing objects created within Emscripten via embind API obey the pass by value and pass by reference convention
\end{itemize}
\end{frame}

\section{Emscripten: Main Module, Side Modules}
\begin{frame}[c, allowframebreaks]\frametitle{Emscripten: Main Module, Side Modules}
  In order to be able to dynamically load a WASM file they need to be compiled with 
  specific options.
  
  Following options can control how Emscripten creates the WebAssembly module:
  \begin{itemize}
    \item \texttt{-s MAIN\_MODULE}
    \item \texttt{-s SIDE\_MODULE}
  \end{itemize}

  The differences between the main module and the side modules:
\framebreak
  \begin{itemize}
    \item Main modules have the system libraries linked in
    \item Side modules are pure wasm files that contain LLVM bitcode that have no system libraries
    \item Side modules rely on the main module linking neede system library functions
    \item Compiling a main module adds a JavaScript file which sets up the Emscripten environment
  \end{itemize}
\end{frame}

\section{Dynamic Loading}
\begin{frame}[allowframebreaks]{Dynamic Loading}
  Three different methods to dynamically link functions:

  \begin{itemize}
    \item dlopen/dlsym: Works the same as with native C code. Requires the side module to be available in the file system. Relies on C-Linkage or using mangled
    C++ names. Obtain dll handle with \texttt{dlopen} and link the functions with \texttt{dlsym}
    \item On startup: Define \texttt{Module.dynamicLibraries = [<list of wasm names>]}. Loads the wasm modules synchronously. Functions from modules need 
    to be declared for compilation to work.
    \item On demand: Use \texttt{EM\_ASM()} macro within C/C++.
  \end{itemize}
  \framebreak

    Using \texttt{loadDynamicLibrary} from within C/C++ seems the best way

    \begin{enumerate}
      \item Loads WASM when really needed
      \item Automatic linking (unlike dlopen)
      \item Asynchronous loading of modules
    \end{enumerate}
\end{frame}

\begin{frame}[fragile, allowframebreaks]{loadDynamicLibrary}
  \begin{itemize}
    \item C++/C function that loads the side modules asynchronously
  \end{itemize}
  \begin{lstlisting}{c++}
void loadLib()
{
  EM_ASM({
    loadDynamicLibrary('output/side_module.wasm', {loadAsync: true, global: true, nodelete: true}).then(
      () => {
        Module.modulePromiseResolve() 
      });
    loadDynamicLibrary('output/side_module2.wasm', {loadAsync: true, global: true, nodelete: true}).then(
      () => {
        Module.module2PromiseResolve()
      });
    loadDynamicLibrary('output/dynamic.wasm', {loadAsync: true, global: true, nodelete: true}).then(
      () => {
        Module.dynamicPromiseResolve()
      });
    });
}
  \end{lstlisting}  

  \framebreak
  \begin{itemize}
    \item Define a promise for each module which the promise in the EM\_ASM call can resolve or reject
  \end{itemize}

  \begin{lstlisting}{JavaScript}
Module.modulePromise = new Promise( (resolve, reject) => {
  Module.modulePromiseResolve = resolve
  Module.modulePromiseReject = reject
});
... // same for module2Pomise and dynamicPromise

Module['onRuntimeInitialized'] = () =>
{
    console.log('onruntimeinitialized');
    Promise.all([Module.modulePromise, Module.module2Promise, Module.dynamicPromise]).then(
      () => {
        console.log("all modules loaded");
        runModuleFunctions();
    })
}
  \end{lstlisting}
\end{frame}



\section{Performance for Calls Between Modules}
\begin{frame}{Performance for Calls Between Modules}
  \begin{itemize}
    \item -O2 for LLVM bc and wasm/js output
    \item embind for JS API
  \end{itemize}
  \url{http://tofino/pgl/WebAssemblyTests}
\end{frame}

\begin{frame}{Notes}
  \begin{itemize}
    \item Development of Dynamic Linking is still ongoing and improvements are still to be expected: \\
    \footnotesize{\url{https://github.com/emscripten-core/emscripten/projects/2}}
  \end{itemize}
\end{frame}
\end{document}